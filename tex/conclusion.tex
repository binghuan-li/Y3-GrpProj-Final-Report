




\section{Conclusion}
\subsection{Summary and Future Work}
\noindent{We have successfully obtained linear calibrations for nitrite, hydrogen peroxide and lactate in their relevant physiological levels in septic patients.\\\\
\noindent{Our sensors could detect nitrite concentrations of 0-96$\mu$M using DPV, with sensitivities of 1.5 to 1.9$n\text{A}/\mu\text{M}$. Hydrogen peroxide ranging over the \textit{in-vivo} concentrations found in healthy individuals (21-36$\mu$M) and septic patients (316$\pm$242.8$\mu$M) were accurately detected \cite{FORMAN201648, VanAsbeck1995}. Nafion modified LOX sensors successfully increase Km from 0.57mM to 1mM to allow detection of up to 4.5mM of lactate with a sensitivity of 5.48mA/mM.}\\\\
Next steps for monitoring these biomarkers involve using the calibration curves obtained from our experiments in serum and fresh blood samples from patients with sepsis and determining  the detection method's accuracy. The biosensor would then be used in large-scale trials to determine the sensitivity, specificity and positive predictive ratio of the biosensor. For the aptamer modelling, future steps would involve testing the NIL model on experimental chronoamperometric data to assess and study the distribution of rate constants for aptamers we test in the lab. The model can be used to extract the value for target concentration more accurately.  Accurate extraction of the rate constants can more accurately measure biomarker concentrations in aptamer biosensors interrogated using chronoamperometry. Lastly, experiments such as those detailed in Appendix part B should be carried out to translate our MN-based biosensor design for in-vivo testing.  
\newpage
    