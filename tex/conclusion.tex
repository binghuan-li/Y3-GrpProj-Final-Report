\section{Conclusion}





\subsection{Summary and Future Work}
\subsubsection{Future of the Microneedle Device}
Our MN biosensor has two main application purposes: (1) accessing the ISF by penetrating the skin barrier and (2) act as a platform for the recognition element.  Hence, more holistic theoretical research has been performed on skin structure and MN design considerations for translation to in-vivo testing in Appendix Part A and B.
\subsubsection{Nitrite}
 Nitrite experimentation has been shown to be repeatable in solution and can accurately determine nitrite concentration. Next steps involve using the calibration curve obtained from these experiments in serum and fresh blood samples from patients with sepsis and determine the accuracy of the detection method.\\\\
 The biosensor would then be used in large-scale trials to determine the sensitivity, specificity and positive predictive ratio of the nitrite biosensor. Improvements to the disposable electrode construction include improving on electrode stability, reducing the need for replacement over time.
\subsubsection{Hydrogen Peroxide}
Immediate future steps for hydrogen peroxide involves testing the electrochemical properties of hydrogen peroxide on disposable electrodes and/or rotating disc electrodes and implementing the electrochemical detection onto the microneedle platform. Meanwhile, an evaluation matrix should be established using calibration result obtained thus far in order for results from real-time monitoring to be well-interpreted, thus, facilitating diagnosis, evolution monitoring and prognosis prediction of sepsis.\\\\
In our study, a calibration curve with gold electrodes was produced over a range of concentrations through chronoamperometry. We have also performed coulometry to validate the theoretical concentrations for each calibration curve. Under extremely low and high hydrogen peroxide concentrations, the percentage difference between the experimental and theoretical concentrations are found to be high. Thus, a discreet consideration is required for future experimental design.
\subsubsection{Lactate}
The sensor is effective at detecting concentrations in the experimented ranges of concentration. Although the values of current plateau in the upper range of lactate concentrations ($>$4.5mM), septic patients normally exhibits levels in the 3.5-4.5mM/L range.\\\\
A Levich study of these lactate sensors would allow us to define parameters regarding mass transport and kinetics around the RDE, including the diffusion coefficient $D$ and standard rate constant $k^{\circ}$, allowing us to compare kinetics of different enzymes. This experiment would involve changing the rotational speed of the RDE and conducting voltammograms \cite{treimer2002consideration}. See \autoref{app:Lactate_future}.
\subsubsection{Aptamer Modelling}
Future work involves testing the numerical inverse laplace model on experimental chronoamperometric data to assess and study the distribution and resolution of rate constants for the aptamers tested in the lab. The model can be used as a tool to perform parameter sensitivity analysis, where we can test the effect of specific variables such as spacer length on the rate constant values. Accurate extraction of the rate constants can allow us to make more realistic chemical and physical models of the aptamers. We have currently ignored the effects of chemical forces (e.g. base to base hydrogen bonding), as well as thermodynamic considerations of the aptamer molecule. More realistic modelling can help us understand the aptamer-target interactions, which will help scientists in this field to better assess the performance of specific aptamers in molecular sensing.
\newpage
