\section{Conclusion}
\subsection{Summary and Future Work}
\subsubsection{Future of the Microneedle Device}
Our MN biosensor has two main application purposes: (1) accessing the ISF by penetrating the skin barrier and (2) act as a platform for the recognition element.  Hence, more holistic theoretical research has been performed on skin structure and MN design considerations for translation to in-vivo testing in Appendix Part A and B.
\subsubsection{Nitrite}
 Nitrite experimentation has been shown to be repeatable in solution and can accurately determine nitrite concentration. Next steps involve using the calibration curve obtained from these experiments in serum and fresh blood samples from patients with sepsis and determine the accuracy of the detection method.\\\\
 The biosensor would then be used in large-scale trials to determine the sensitivity, specificity and positive predictive ratio of the nitrite biosensor.
\subsubsection{Hydrogen Peroxide}
% following part the conclusion
In this study, a hydrogen peroxide calibration curve with gold electrodes was produced with the concentrations from 0 to 300uM, which has ranged over the \textit{in vivo}  concentrations found in both the healthy individuals and the septic patients \cite{FORMAN201648, VanAsbeck1995}. Following that, we have validated the experimental concentrations by coulometry, finding that a consistency was achieved between the experimental and theoretical concentrations. \\\\
% following part is the furture work
Immediate future steps for hydrogen peroxide involves testing the electrochemical properties of hydrogen peroxide on disposable electrodes and/or rotating disc electrodes and implementing the electrochemical detection onto the microneedle detection platform. Meanwhile, an evaluation matrix to interpret the real-time monitoring result should be established based on the calibration result, correspondingly.

\subsubsection{Lactate}
The Nafion modified sensor is effective at increasing the effective Km value of our LOX, successfully optimising the detection range of the sensors to our target range. Although the values of current begin to plateau at higher concentration ($>$4.0mM), the sensor at least detects changes up until the start of the septic range.\\\\
A Levich study of these lactate sensors would allow us to define parameters regarding mass transport and kinetics around the RDE, allowing us to compare detection with different modifications to electrode design \autoref{app:Lactate_future}.
\subsubsection{Aptamer Modelling}
Future work involves testing the numerical inverse laplace model on experimental chronoamperometric data to assess and study the distribution of rate constants for aptamers we test in the lab. The model can be used as a tool to more accurately extract the value for target concentration. Accurate extraction of the rate constants can allow us to make more realistic chemical and physical models of the aptamers.
\newpage
