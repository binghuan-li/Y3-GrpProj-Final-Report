\section{Introduction}
\subsection{Project Aim}
At a time when demand for healthcare is rapidly increasing, healthcare systems globally are struggling with depleting resources.  However, early accurate diagnosis of sepsis is vital as each hour of delay in initiating antibiotic therapy increases mortality by 5–10$\%$ \cite{heikenfeld2018lab,bauer2010molecular}. Diagnosis of sepsis is not straightforward and interpreting clinical signs and symptoms is a challenge resulting in a potential delay in diagnosis. Uncertainty about diagnosis is mitigated by needless antimicrobial therapy or the use of broad-spectrum antibiotics but this gives rise to antibiotic resistance\cite{hung2020current}. Currently, there is no gold standard for diagnosing sepsis and constant monitoring by medical staff is required. Therefore, our research aims to develop an on-body portable and wearable biosensing point-of-care (POC) microneedle (MN) device for infection diagnostics that can be self-managed by the public and is capable of real-time detection with high specificity and sensitivity. The device detects sepsis biomarkers in skin interstitial (ISF) as it mirrors blood composition and allows the device to be tiny avoiding contact with nerve endings and blood capillaries, thus, making it toleratable by patients as it provides a painless user experience \cite{liu2020microneedles}. \autoref{Appendix_A} provides a more comprehensive evaluation for utilising skin ISF as the biofluid source.\\\\
Our project focuses on electrochemically detecting different sepsis biomarkers: 1) Nitrite, 2) Hydrogen Peroxide, and 3) Lactate. We studied a panel of sepsis biomarkers in order to diagnose sepsis with high specificity. In addition, this project aims to develop more representative models for aptamer-based biosensors interrogated using chronoamperometry. This method of detection allows drift-free, sub-second resolved measurements of the target molecule \cite{arroyo2018subsecond}. More accurate and representative modelling of the aptamers and their output signals will improve the precision of biomarker measurements.\\\\